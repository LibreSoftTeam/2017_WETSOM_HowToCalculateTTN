
%% bare_conf.tex
%% V1.3
%% 2007/01/11
%% by Michael Shell
%% See:
%% http://www.michaelshell.org/
%% for current contact information.
%%
%% This is a skeleton file demonstrating the use of IEEEtran.cls
%% (requires IEEEtran.cls version 1.7 or later) with an IEEE conference paper.
%%
%% Support sites:
%% http://www.michaelshell.org/tex/ieeetran/
%% http://www.ctan.org/tex-archive/macros/latex/contrib/IEEEtran/
%% and
%% http://www.ieee.org/

%%*************************************************************************
%% Legal Notice:
%% This code is offered as-is without any warranty either expressed or
%% implied; without even the implied warranty of MERCHANTABILITY or
%% FITNESS FOR A PARTICULAR PURPOSE! 
%% User assumes all risk.
%% In no event shall IEEE or any contributor to this code be liable for
%% any damages or losses, including, but not limited to, incidental,
%% consequential, or any other damages, resulting from the use or misuse
%% of any information contained here.
%%
%% All comments are the opinions of their respective authors and are not
%% necessarily endorsed by the IEEE.
%%
%% This work is distributed under the LaTeX Project Public License (LPPL)
%% ( http://www.latex-project.org/ ) version 1.3, and may be freely used,
%% distributed and modified. A copy of the LPPL, version 1.3, is included
%% in the base LaTeX documentation of all distributions of LaTeX released
%% 2003/12/01 or later.
%% Retain all contribution notices and credits.
%% ** Modified files should be clearly indicated as such, including  **
%% ** renaming them and changing author support contact information. **
%%
%% File list of work: IEEEtran.cls, IEEEtran_HOWTO.pdf, bare_adv.tex,
%%                    bare_conf.tex, bare_jrnl.tex, bare_jrnl_compsoc.tex
%%*************************************************************************

% *** Authors should verify (and, if needed, correct) their LaTeX system  ***
% *** with the testflow diagnostic prior to trusting their LaTeX platform ***
% *** with production work. IEEE's font choices can trigger bugs that do  ***
% *** not appear when using other class files.                            ***
% The testflow support page is at:
% http://www.michaelshell.org/tex/testflow/



% Note that the a4paper option is mainly intended so that authors in
% countries using A4 can easily print to A4 and see how their papers will
% look in print - the typesetting of the document will not typically be
% affected with changes in paper size (but the bottom and side margins will).
% Use the testflow package mentioned above to verify correct handling of
% both paper sizes by the user's LaTeX system.
%
% Also note that the "draftcls" or "draftclsnofoot", not "draft", option
% should be used if it is desired that the figures are to be displayed in
% draft mode.
%
\documentclass[10pt, conference]{IEEEtran}
% Add the compsocconf option for Computer Society conferences.
%
% If IEEEtran.cls has not been installed into the LaTeX system files,
% manually specify the path to it like:
% \documentclass[conference]{../sty/IEEEtran}





% Some very useful LaTeX packages include:
% (uncomment the ones you want to load)


% *** MISC UTILITY PACKAGES ***
%
%\usepackage{ifpdf}
% Heiko Oberdiek's ifpdf.sty is very useful if you need conditional
% compilation based on whether the output is pdf or dvi.
% usage:
% \ifpdf
%   % pdf code
% \else
%   % dvi code
% \fi
% The latest version of ifpdf.sty can be obtained from:
% http://www.ctan.org/tex-archive/macros/latex/contrib/oberdiek/
% Also, note that IEEEtran.cls V1.7 and later provides a builtin
% \ifCLASSINFOpdf conditional that works the same way.
% When switching from latex to pdflatex and vice-versa, the compiler may
% have to be run twice to clear warning/error messages.






% *** CITATION PACKAGES ***
%
%\usepackage{cite}
% cite.sty was written by Donald Arseneau
% V1.6 and later of IEEEtran pre-defines the format of the cite.sty package
% \cite{} output to follow that of IEEE. Loading the cite package will
% result in citation numbers being automatically sorted and properly
% "compressed/ranged". e.g., [1], [9], [2], [7], [5], [6] without using
% cite.sty will become [1], [2], [5]--[7], [9] using cite.sty. cite.sty's
% \cite will automatically add leading space, if needed. Use cite.sty's
% noadjust option (cite.sty V3.8 and later) if you want to turn this off.
% cite.sty is already installed on most LaTeX systems. Be sure and use
% version 4.0 (2003-05-27) and later if using hyperref.sty. cite.sty does
% not currently provide for hyperlinked citations.
% The latest version can be obtained at:
% http://www.ctan.org/tex-archive/macros/latex/contrib/cite/
% The documentation is contained in the cite.sty file itself.






% *** GRAPHICS RELATED PACKAGES ***
%
\ifCLASSINFOpdf
  % \usepackage[pdftex]{graphicx}
  % declare the path(s) where your graphic files are
  % \graphicspath{{../pdf/}{../jpeg/}}
  % and their extensions so you won't have to specify these with
  % every instance of \includegraphics
  % \DeclareGraphicsExtensions{.pdf,.jpeg,.png}
\else
  % or other class option (dvipsone, dvipdf, if not using dvips). graphicx
  % will default to the driver specified in the system graphics.cfg if no
  % driver is specified.
  % \usepackage[dvips]{graphicx}
  % declare the path(s) where your graphic files are
  % \graphicspath{{../eps/}}
  % and their extensions so you won't have to specify these with
  % every instance of \includegraphics
  % \DeclareGraphicsExtensions{.eps}
\fi
% graphicx was written by David Carlisle and Sebastian Rahtz. It is
% required if you want graphics, photos, etc. graphicx.sty is already
% installed on most LaTeX systems. The latest version and documentation can
% be obtained at: 
% http://www.ctan.org/tex-archive/macros/latex/required/graphics/
% Another good source of documentation is "Using Imported Graphics in
% LaTeX2e" by Keith Reckdahl which can be found as epslatex.ps or
% epslatex.pdf at: http://www.ctan.org/tex-archive/info/
%
% latex, and pdflatex in dvi mode, support graphics in encapsulated
% postscript (.eps) format. pdflatex in pdf mode supports graphics
% in .pdf, .jpeg, .png and .mps (metapost) formats. Users should ensure
% that all non-photo figures use a vector format (.eps, .pdf, .mps) and
% not a bitmapped formats (.jpeg, .png). IEEE frowns on bitmapped formats
% which can result in "jaggedy"/blurry rendering of lines and letters as
% well as large increases in file sizes.
%
% You can find documentation about the pdfTeX application at:
% http://www.tug.org/applications/pdftex





% *** MATH PACKAGES ***
%
%\usepackage[cmex10]{amsmath}
% A popular package from the American Mathematical Society that provides
% many useful and powerful commands for dealing with mathematics. If using
% it, be sure to load this package with the cmex10 option to ensure that
% only type 1 fonts will utilized at all point sizes. Without this option,
% it is possible that some math symbols, particularly those within
% footnotes, will be rendered in bitmap form which will result in a
% document that can not be IEEE Xplore compliant!
%
% Also, note that the amsmath package sets \interdisplaylinepenalty to 10000
% thus preventing page breaks from occurring within multiline equations. Use:
%\interdisplaylinepenalty=2500
% after loading amsmath to restore such page breaks as IEEEtran.cls normally
% does. amsmath.sty is already installed on most LaTeX systems. The latest
% version and documentation can be obtained at:
% http://www.ctan.org/tex-archive/macros/latex/required/amslatex/math/





% *** SPECIALIZED LIST PACKAGES ***
%
%\usepackage{algorithmic}
% algorithmic.sty was written by Peter Williams and Rogerio Brito.
% This package provides an algorithmic environment fo describing algorithms.
% You can use the algorithmic environment in-text or within a figure
% environment to provide for a floating algorithm. Do NOT use the algorithm
% floating environment provided by algorithm.sty (by the same authors) or
% algorithm2e.sty (by Christophe Fiorio) as IEEE does not use dedicated
% algorithm float types and packages that provide these will not provide
% correct IEEE style captions. The latest version and documentation of
% algorithmic.sty can be obtained at:
% http://www.ctan.org/tex-archive/macros/latex/contrib/algorithms/
% There is also a support site at:
% http://algorithms.berlios.de/index.html
% Also of interest may be the (relatively newer and more customizable)
% algorithmicx.sty package by Szasz Janos:
% http://www.ctan.org/tex-archive/macros/latex/contrib/algorithmicx/
\usepackage{amsmath}
\usepackage{centernot}
\usepackage{tikz}
\usepackage{xspace}

% *** ALIGNMENT PACKAGES ***
%
%\usepackage{array}
% Frank Mittelbach's and David Carlisle's array.sty patches and improves
% the standard LaTeX2e array and tabular environments to provide better
% appearance and additional user controls. As the default LaTeX2e table
% generation code is lacking to the point of almost being broken with
% respect to the quality of the end results, all users are strongly
% advised to use an enhanced (at the very least that provided by array.sty)
% set of table tools. array.sty is already installed on most systems. The
% latest version and documentation can be obtained at:
% http://www.ctan.org/tex-archive/macros/latex/required/tools/


%\usepackage{mdwmath}
%\usepackage{mdwtab}
% Also highly recommended is Mark Wooding's extremely powerful MDW tools,
% especially mdwmath.sty and mdwtab.sty which are used to format equations
% and tables, respectively. The MDWtools set is already installed on most
% LaTeX systems. The lastest version and documentation is available at:
% http://www.ctan.org/tex-archive/macros/latex/contrib/mdwtools/


% IEEEtran contains the IEEEeqnarray family of commands that can be used to
% generate multiline equations as well as matrices, tables, etc., of high
% quality.


%\usepackage{eqparbox}
% Also of notable interest is Scott Pakin's eqparbox package for creating
% (automatically sized) equal width boxes - aka "natural width parboxes".
% Available at:
% http://www.ctan.org/tex-archive/macros/latex/contrib/eqparbox/





% *** SUBFIGURE PACKAGES ***
%\usepackage[tight,footnotesize]{subfigure}
% subfigure.sty was written by Steven Douglas Cochran. This package makes it
% easy to put subfigures in your figures. e.g., "Figure 1a and 1b". For IEEE
% work, it is a good idea to load it with the tight package option to reduce
% the amount of white space around the subfigures. subfigure.sty is already
% installed on most LaTeX systems. The latest version and documentation can
% be obtained at:
% http://www.ctan.org/tex-archive/obsolete/macros/latex/contrib/subfigure/
% subfigure.sty has been superceeded by subfig.sty.



%\usepackage[caption=false]{caption}
%\usepackage[font=footnotesize]{subfig}
% subfig.sty, also written by Steven Douglas Cochran, is the modern
% replacement for subfigure.sty. However, subfig.sty requires and
% automatically loads Axel Sommerfeldt's caption.sty which will override
% IEEEtran.cls handling of captions and this will result in nonIEEE style
% figure/table captions. To prevent this problem, be sure and preload
% caption.sty with its "caption=false" package option. This is will preserve
% IEEEtran.cls handing of captions. Version 1.3 (2005/06/28) and later 
% (recommended due to many improvements over 1.2) of subfig.sty supports
% the caption=false option directly:
%\usepackage[caption=false,font=footnotesize]{subfig}
%
% The latest version and documentation can be obtained at:
% http://www.ctan.org/tex-archive/macros/latex/contrib/subfig/
% The latest version and documentation of caption.sty can be obtained at:
% http://www.ctan.org/tex-archive/macros/latex/contrib/caption/




% *** FLOAT PACKAGES ***
%
%\usepackage{fixltx2e}
% fixltx2e, the successor to the earlier fix2col.sty, was written by
% Frank Mittelbach and David Carlisle. This package corrects a few problems
% in the LaTeX2e kernel, the most notable of which is that in current
% LaTeX2e releases, the ordering of single and double column floats is not
% guaranteed to be preserved. Thus, an unpatched LaTeX2e can allow a
% single column figure to be placed prior to an earlier double column
% figure. The latest version and documentation can be found at:
% http://www.ctan.org/tex-archive/macros/latex/base/



%\usepackage{stfloats}
% stfloats.sty was written by Sigitas Tolusis. This package gives LaTeX2e
% the ability to do double column floats at the bottom of the page as well
% as the top. (e.g., "\begin{figure*}[!b]" is not normally possible in
% LaTeX2e). It also provides a command:
%\fnbelowfloat
% to enable the placement of footnotes below bottom floats (the standard
% LaTeX2e kernel puts them above bottom floats). This is an invasive package
% which rewrites many portions of the LaTeX2e float routines. It may not work
% with other packages that modify the LaTeX2e float routines. The latest
% version and documentation can be obtained at:
% http://www.ctan.org/tex-archive/macros/latex/contrib/sttools/
% Documentation is contained in the stfloats.sty comments as well as in the
% presfull.pdf file. Do not use the stfloats baselinefloat ability as IEEE
% does not allow \baselineskip to stretch. Authors submitting work to the
% IEEE should note that IEEE rarely uses double column equations and
% that authors should try to avoid such use. Do not be tempted to use the
% cuted.sty or midfloat.sty packages (also by Sigitas Tolusis) as IEEE does
% not format its papers in such ways.

\newif\ifdraft
\drafttrue

\input{macros}



% *** PDF, URL AND HYPERLINK PACKAGES ***
%
\usepackage{url}
% url.sty was written by Donald Arseneau. It provides better support for
% handling and breaking URLs. url.sty is already installed on most LaTeX
% systems. The latest version can be obtained at:
% http://www.ctan.org/tex-archive/macros/latex/contrib/misc/
% Read the url.sty source comments for usage information. Basically,
% \url{my_url_here}.





% *** Do not adjust lengths that control margins, column widths, etc. ***
% *** Do not use packages that alter fonts (such as pslatex).         ***
% There should be no need to do such things with IEEEtran.cls V1.6 and later.
% (Unless specifically asked to do so by the journal or conference you plan
% to submit to, of course. )


% correct bad hyphenation here
\hyphenation{op-tical net-works semi-conduc-tor}


\begin{document}
%
% paper title
% can use linebreaks \\ within to get better formatting as desired
\title{How much time did it take to notify a Bug? \\ Two case of studies: ElasticSearch and Nova}


% author names and affiliations
% use a multiple column layout for up to two different
% affiliations

\author{\IEEEauthorblockN{Gema Rodriguez-Perez}
\IEEEauthorblockA{LibreSoft/GSyC\\
Universidad Rey Juan Carlos\\
Fuenlabrada, Spain\\
gerope@libresoft.es}
\and
\IEEEauthorblockN{Jesus M. Gonzalez-Barahona}
\IEEEauthorblockA{LibreSoft/GSyC\\
Universidad Rey Juan Carlos\\
Fuenlabrada, Spain\\
jgb@gsyc.es}
\and
\IEEEauthorblockN{Gregorio Robles}
\IEEEauthorblockA{LibreSoft/GSyC\\
Universidad Rey Juan Carlos\\
Fuenlabrada, Spain\\
grex@gsyc.urjc.es}
}

% conference papers do not typically use \thanks and this command
% is locked out in conference mode. If really needed, such as for
% the acknowledgment of grants, issue a \IEEEoverridecommandlockouts
% after \documentclass

% for over three affiliations, or if they all won't fit within the width
% of the page, use this alternative format:
% 
%\author{\IEEEauthorblockN{Michael Shell\IEEEauthorrefmark{1},
%Homer Simpson\IEEEauthorrefmark{2},
%James Kirk\IEEEauthorrefmark{3}, 
%Montgomery Scott\IEEEauthorrefmark{3} and
%Eldon Tyrell\IEEEauthorrefmark{4}}
%\IEEEauthorblockA{\IEEEauthorrefmark{1}School of Electrical and Computer Engineering\\
%Georgia Institute of Technology,
%Atlanta, Georgia 30332--0250\\ Email: see http://www.michaelshell.org/contact.html}
%\IEEEauthorblockA{\IEEEauthorrefmark{2}Twentieth Century Fox, Springfield, USA\\
%Email: homer@thesimpsons.com}
%\IEEEauthorblockA{\IEEEauthorrefmark{3}Starfleet Academy, San Francisco, California 96678-2391\\
%Telephone: (800) 555--1212, Fax: (888) 555--1212}
%\IEEEauthorblockA{\IEEEauthorrefmark{4}Tyrell Inc., 123 Replicant Street, Los Angeles, California 90210--4321}}




% use for special paper notices
%\IEEEspecialpapernotice{(Invited Paper)}




% make the title area
\maketitle


\begin{abstract}

The \emph{Time To Notify} (TTN) a bug is a valuable metric in the software maintenance and evolution studies that describes how much time it takes for a bug to be notified/reported in the issue tracking system since the time the bug was introduced into the source code. Even so, it is still a challenge to exactly calculate it since no precise way exists to locate the line where the bug originated. This paper aims to study what is the value of TTN in a two different projects where we know exactly which ``previous commit'' was the cause of the failure in the system. Furthermore, to better understand how this is related to the maintenance and evolution of a software, we also analyze the relationship between TTN and other metrics extracted from the source code management (SCM) system such as the author of the bug, the \emph{time to fix} (TTF) or the developer experience.\\ 


As results, we have observed that the mean of the TTN in the projects was 312 days and 431 days. However, only one of the projects showed a weak correlation between the experience of the author who created the bug and TTN.

\end{abstract}

\begin{IEEEkeywords}
Bug introduction change; bug seeding metrics;

\end{IEEEkeywords}


% For peer review papers, you can put extra information on the cover
% page as needed:
% \ifCLASSOPTIONpeerreview
% \begin{center} \bfseries EDICS Category: 3-BBND \end{center}
% \fi
%
% For peerreview papers, this IEEEtran command inserts a page break and
% creates the second title. It will be ignored for other modes.
\IEEEpeerreviewmaketitle



\section{Introduction}
\label{sec:intro}

The concept of \emph{Time To Notify} a bug is really useful since measure the time from the buggy line was introduced in the code until some developer reported the wrong behavior of the software. \grex{It is better to not repeat sentences from the abstract} \gema{Is it better now? Or I have to change completely the beginning of the paragraph? } TTN could be computed as part of the bug fixing activity, which is one of the core activities in software maintenance and evolution. Nonetheless, it is estimated that 80\% of the total cost of a software system is spent in maintenance and evolution~\cite{tassey2002economic}. Hence, researchers are spending much effort understanding and characterizing software maintenance and evolution processes with metrics such as the \emph{Time To Fix} (TTF), the \emph{Time To Review} (TTR) or the \emph{fix-effort} to improve code quality.

In fact, to compute the life of a bug in a software, it is necessary know exactly in what line of source code the bug was introduced. As in modern software development, versioning systems such as git are commonly used, meta-data (committer, author, time, etc.) to when the line of source code was included can be retrieved. However, finding the line of code where the bug was introduce is no not a trivial task; several methods can be used to determine what commits are the candidates of inserting a bug. The popular SZZ algorithm~\cite{sliwerski2005changes} is one of them: it traces the lines touched in a \emph{fixing} commit back to the time when these lines were modified or added. The main concern in using this algorithm is the flawed metrics it may provided in some specific scenarios that frequently happen during the evolution of a sofware system. This scenarios such as changes in the API or update versions present a common characteristic, the previous commit(s) cannot be blamed as the cause of inserting the bug due to these lines were correct at the time of insertion. \grex{Not clear what you want to say in the last sentence}. \grex{We should say that SZZ provides a \emph{flawed} metric, because it does not consider specific scenarios that frequently happen during the evolution of a software system, such as changes in the API, etc.} \gema{Have I to clarify more the issue?}
%The resulting commit or set of commits is considered as the one who introduced the bug. Thus, all the metrics calculated using such method present an estimation value of the real time because they are based in this assumption where the bug should be in the previous modifications of the fixed lines.

In this paper, we measure TTN for a set of bugs from two different Free/Libre/Open Source (FLOSS) software projects, ElasticSearch and Nova. Because of previous research performed on them, we know exactly the location of the \emph{Bug Introducing Change} (BIC) for a set of bugs. This means that from a bunch of ``previous commits'' of each of the fixing lines, we are able to identify what specific previous commit or set of previous commits caused the failure in the system.

Furthermore, to better understand how this is related to the maintenance and evolution of a software, we also analyze the relationship between TTN and other metrics extracted from the source code management (SCM) system such as the author of the bug, the time to fix or the developer experience inserting the bug.

Summing up, we are interested in studying the real value of TTN to answer following research questions:

\begin{itemize}
\item RQ1: What are the real values of TTN in both projects? What is its mean, median, etc.?
\item RQ2: Is there any correlation of TTN with others measures in the bug fixing process?
\end{itemize}

Our main findings reveal that the \emph{real} mean TTN ranges from ten months in ElasticSearch to fourteen months in Nova. \grex{How different is it from the way it is done with SZZ?} \gema{If we assume the optimistic cases the mean in Nova is eight months and in ElasticSearch five months. On the contrary, if we assume the pessimistic cases, the mean in Nova is thirteen months and twelve months in ElastichSearch}. On the other hand, only one of the projects showed a weak correlation between TTN and the experience of the author who introduced the bug.

The remainder of the paper is organized as follows: In Section~\ref{sec:relatedwork} the current body of knowledge is presented. Next, Section~\ref{sec:methodology} describes the methodology used to identify the Bug Introducing Change and calculates the metrics, followed by the results obtained from Nova and ElasticSearch in Section~\ref{sec:results}. Section~\ref{sec:discussion} answers the research questions and discusses potential applications and improvements to our approach. After reporting the limitations and threats to validity in Section~\ref{sec:threats}, we draw some conclusions and point out potential future work in Section~\ref{sec:conclusions}.

\section{Related Work}
\label{sec:relatedwork}

To measure experience exists several ways: (1) Number of Commits~\cite{eyolfson2011time}, (2) Fixing activity~\cite{ahsan2010mining} and (3) Ownership~\cite{german2004using}

I must talk about these prior studies:
 Metrics for Gerrit code reviews~\cite{lehtonen2015metrics} --> Monitoring the code review process is not as easy as was we could think in spite of tools as Gerrit, they introduced the Review time, integration time, average number of patch-sets (Maybe discard it)???
 
Studying the Fix-Time for Bugs in Large Open Source Projects ~\cite{marks2011studying} --> They studied the dependence of  the fix-time for bugs with their attributes in a large open source projects where the results indicated that the priority of the bug in Eclipse was correlated with the time to fix

How long did it take to fix bugs?~\cite{kim2006long} --> They have calculated this time but the name was bug-fix time ( It requires much effort to fix bugs,e.g.Kim and Whitehead [14] report that the median time for fixing a single bug is about 200 days.)(studied  the  life  span  of  bugs  in  ArgoUML  and  PostgreSQL projects, and found that bug-fixing time had a median of about 200 days)

Bug-fix Time Prediction Models: Can We Do Better? ~\cite{bhattacharya2011bug} 

Do More Experienced Developers Introduce Fewer Bugs? ~\cite{izquierdo2012more} --> they analyzed some Mozilla modules expecting statistical differences between developers with different levels of experience and the introduction of bugs, but the results didn't show that correlation. More experience doesn't imply less bug introducing changes. They used the SZZ

Are developers fixing their own bugs?: Tracing bug-fixing and bug-seeding committers ~\cite{izquierdo2011developers} --> They were looking to understand if these contributors who introduced the bug are the same who fixed it, the results showed that in most cases the bug fixing activity was not carried out by the author or the bug introducing change. assume SZZ theory

Talk about how different is our study to prior works:

Prior studies~\cite{kim2006long} have proposed the \emph{Bug Fixing Time} (BFT), it's a similar metric to TTN but while the \emph{Bug Fixing Time} is an estimation to calculate the time since bugs were introduced until the bug was fixed, the TTN is a concrete metric to compute the time since bugs were injected until they notified the wrong behaviour of the software, this value doesn't have include the time to fix.

\section{Methodology}
\label{sec:methodology}

In our study the data analyzed was obtained from the source code management, the issue tracking system and the code review system. We will illustrate our methodology in Figure~\ref{fig:methodology} where the income is a set of bug reports extracted randomly from the issue tracking system. The steps of our process are given by white boxes; the colored boxes give the sources that are used to obtain the information required in each step.

\begin{figure}[ht]
\centering
\includegraphics[width=\columnwidth]{methodology.png}
\caption{The methodology diagram, starting with the analysis of a fix commit and finishing with the value of the metrics}
\label{fig:methodology}       % Give a unique label
\end{figure}

The following list is a detailed description of the steps: 

\begin{enumerate}
		\item Find the fixing commit of each bug report.
		\item Find the lines that this commit added, modified or deleted to solve the bug.
		\item Obtain, for each of those lines, the commit that added, modified or deleted these lines previously. The result is the previous commit for each line.
		\item Analyze which one of the previous commits was the BIC. In the case where none of them was the cause of the bug, track back each previous commit until the line causing the failure can be found in some of the previous commits. \grex{What happens if it is a fixing commit that only adds lines?} \gema{In the case of only new lines were added to the fixing commit, we analyzed the nearby commits to those lines looking whether some of them forgot to add this lines.}
		\item Extract the date when the fixing commit, the BIC and the bug report was submitted, as well as the author of each of the commits to calculate our metrics.	
\end{enumerate} 


\subsection{Metrics}

To compute all the metrics used in this study, we need to identify the BIC. Once we have it, we are able to measure exactly all the time values we want.  

Next, we describe the metrics. Notice that the value of all these metrics has been calculated in days: 

\begin{itemize}
		\item \emph{Experience Until the Bug} (EUB): The time of experience that presents the author of the bug introducing change at the moment of submit such commit to the repository. The experience is measured from the first time that the author committed some code to the project until (s)he introduced the commit with the buggy line.
		\item \emph{Time To Notify} (TTN): This value measures the time since the commit inserted the bug was merged into he master branch until some developer notified the unexpected behavior and reported it. 
		\item \emph{Time To Fix} (TTF): It refers to the necessary time to fix a bug. This value is the period from a bug report was opened until it was closed by a fixing commit.
		\item \emph{Bug Fixing Time} (BFT): It measures the time from the Bug Inserting change date and the fixing commit date. This time is calculated as the addition of the TTN and TTF time.
\end{itemize} 

The Figure~\ref{fig:metrics} shows when the metric under interest begins and ends. and their corresponding fixes to be able of measure the time between them. (Example)

\begin{figure}[ht]
\centering
\includegraphics[width=\columnwidth]{metrics.png}
\caption{ Span period of the metrics: Experience Until Bug (EUB),  Time To Notify (TTN), Time To Fix (TTF), Bug Fixing Time (BFT) }
\label{fig:metrics}       % Give a unique label
\end{figure}

Besides this metrics, we calculate two more metrics related with the Time To Notify a bug which will be used to compare our results with the ones getting after using the SZZ algorithm. Until now, all the values have been calculated using the exactly location of the bug introducing change, but the SZZ algorithm does not locate the precise commit in which the bug was inserted, and also its outcome could be a a set of commits among which the interest commit may or may not be.

To compare how the TTN obtained through SZZ algorithm differ to ours, we describe this two times:
\begin{itemize}
	\item \emph{Time To Notify Optimistic} (TTN\_o): It assumes that the bug introducing change was the nearest commit to the bug report date, thus it measures the time has past since such commit was done until the wrong behavior was reported.
	\item \emph{Time To Notify Pessimistic} (TTN\_p): It assumes that the furthest commit was the bug introducing change, thus it measures the worst time possible until notify the bug.
\end{itemize}

\section{Evaluation}
\label{sec:evaluation}
We have validated our methodology analyzing the tickets from two different projects written in different programming languages. Nova uses a dynamic language whereas ElasticSearch uses a statically language. Thus, we may study the dependency of the Time To Notify with the programming language that the project is using.

In one hand, Nova belongs to OpenStack project which is a cloud computing platform with a huge developing community (more than 5,000 developers) and significant industrial support from several major companies such as Red Hat, Intel, IBM, HP, etc. The source code of Nova is written in Python and was particularly of interest because it is continuously evolving due to its very active community. Currently it has more than 44,000 commits with more than 2 million lines of code and around 1,000 contributors \footnote{\url{http://activity.openstack.org/dash/browser/repository.html?repository=nova.git&ds=scm}}. All its code is hosted and available at GitHub\footnote{\url{https://github.com/openstack/nova}} and it works with Git\footnote{\url{https://git-scm.com/}} as the source code management, Launchpad\footnote{\url{https://launchpad.net/}} as the issue tracking system  and  Gerrit\footnote{\url{https://www.gerritcodereview.com/}} as the code review system.

On the contrary, ElasticSearch is a distributed open source search and analytics engine written in Java. In spite of it has less number of commits and contributors than OpenStack, 26,000 commit and 764 contributors, its policy of label an issue as bug report is very strict, thus we could be sure that the tickets analyzed are real bug reports. Furthermore, all its code is hosted at GitHub\footnote{\url{https://github.com/elastic/elasticsearch/}} and it  has been used such as issue tracking system and pull request system in this project.

In this study we used a data set created in a prior research work where we analyzed manually and in detail a total of 76 bug fixing commits of real bug reports, looking in each one for the bug introducing change that caused the failure. Thus, once we have the commit that induced the later fix, we calculate the values of the TTN, EUB, TTF and the authors of the fixing commit, the bug report, and the bug introducing change.  

\section{Results}
\label{sec:results}

We have analyzed exactly the values of the time to notify, the experience of the author until inserted the bug, and the time to fix in the 76 bug fixing commits, 39 belong to Nova and 37 belong to ElasticSearch.  In addition, we have presented such values with the optimistic and pessimistic values of the time to notify when the SZZ algorithm was used in the analysis.

The Table~\ref{table} shows the mean of the Time To Fix value and the Time To Notify values computed in each project using the exactly bug introducing change or, on the contrary, using the best and worst suspicious commit to insert the bug by SZZ.  
\begin{table}[!t]
% increase table row spacing, adjust to taste
\renewcommand{\arraystretch}{1.3}
\label{table}
\centering
\caption{Mean in days of the TTN, TTN\_o, TTN\_p and TTF in each project}
% Some packages, such as MDW tools, offer better commands for making tables
%% than the plain LaTeX2e tabular which is used here.
\begin{tabular}{|c||c||c||c||c| }
\hline
  & TTN & TTN\_o & TTN\_p & TTF \\
\hline
Nova & 431 & 257 & 410 & 64 \\
\hline
ElasticSearch & 312 & 171 & 375 & 14\\
\hline
\end{tabular}
\end{table}

The Figure~\ref{fig:meansOfNova} and Figure~\ref{fig:meansOfES} shows the TTN as well as the TTF and the EUB in both projects using box plots. For example, to notify 50\% of the bugs requires approximately 100 to 600 days in both projects. Whereas the median is 300 days in Nova and almost 200 days in ElasticSearch.  

\begin{figure}[ht]
\centering
\includegraphics[width=\columnwidth]{boxplotNova.png}
<<<<<<< HEAD
\caption{The boxes indicate 50\% of TTN, TTN\_o, TTN\_p, TTF and EUB (25\% to 75\% quartile) in Nova project. The middle line in boxes indicates the median value}
\label{fig:meansOfNova}       % Give a unique label
=======
\caption{Box-Plot in Nova}
\label{fig:graph4}       % Give a unique label
>>>>>>> 2c71d801f85fe5d6a75793ce2c15c0f9f3bc2eee
\end{figure}

\begin{figure}[ht]
\centering
\includegraphics[width=\columnwidth]{boxplotES.png}
<<<<<<< HEAD
\caption{The boxes indicate 50\% of TTN, TTN\_o, TTN\_p, TTF and EUB (25\% to 75\% quartile) in ElasticSearch project. The middle line in boxes indicates the median value}
\label{fig:meansOfES}       % Give a unique label
=======
\caption{Box-Plot in ES}
\label{fig:graph5}       % Give a unique label
>>>>>>> 2c71d801f85fe5d6a75793ce2c15c0f9f3bc2eee
\end{figure}

The Figure~\ref{fig:correlation} shows the correlation between each of the metrics analyzed in this study in the Nova project. It shows a strong positive correlation between the TTN, TTN\_o and TTN\_p, also it shows a weak negative correlation between the experience of the author introducing the bug and the time to notify. The ElasticSearch project only showed the strong positive correlation between the TTN variables, and we decided not display the graph.

\begin{figure}[ht]
\centering
\includegraphics[width=\columnwidth]{correlationMatrix.png}
\caption{Correlation Matrix of each metric in Nova project}
\label{fig:correlation}       % Give a unique label
\end{figure}

- Describe these plots
\begin{figure}[ht]
\centering
\includegraphics[width=\columnwidth]{DistributionNova_b.png}
\caption{Distribution of the metrics in Nova project}
\label{fig:graph2}       % Give a unique label
\end{figure}

\begin{figure}[ht]
\centering
\includegraphics[width=\columnwidth]{DistributionES_b.png}
\caption{Distribution of the metrics in Nova project}
\label{fig:graph3}       % Give a unique label
\end{figure}


\section{Discussion}
\label{sec:discussion}
- Possibles explanations for the TTN according to the results and have into account the other metrics


\section{Threats to validity}
\label{sec:threats}

The limited sample size of tickets used in this research is the major threat to its validity. It is relatively high considering that the analysis was done manually and in detail to reach reliable results, but there is a long way to get a representative sample from a variety of free/open source systems, or software projects at large. Our analysis requires a lot of human effort, so increasing meaningfully the number of tickets is difficult. However, it should be noted that our numbers are the order of magnitude of similar studies: for instance Hindle's \emph{et al.} ~\cite{hindle2008large} article on large commits, considered 100 commits.

Other internal threats to validity are:
\begin{itemize}
    \item We are only using part of the information that the tickets provide, like comments and text. There could be some patterns that can be found in other parts of the information, but we think that this information is enough to determine where the bug introducing change is, due to the comments of the developers in the ticket give several information about what is failing in the code.
    \item In some cases, when only new code was added by the bug fixing commit, researchers may have some problems to compare their TTN with the SZZ because of SZZ does not considered the new additions in the fixing commits. 
    \item The experience of the developer calculated from its first commit in the system may not be the best definition to indicate the experience of the author due to a old contributor could have less commit and as a consequence less experience in the project that a new active contributor who is committing all time.
\end{itemize}

The most important external threats, most of them related to peculiarities of the Nova project, are:
\begin{itemize}
    \item OpenStack is a special project with a very rapid evolution, and a very active community of developer and ElasticSearch is relatively new project with a strong criteria in the bug fix activity. Maybe, in other projects with less commits per year, results may be totally different.
    \item The programming languages analyzed at this research are Python and Java Script: It may happen that other programming languages present different results
\end{itemize}

\section{Conclusion}
\label{sec:conclusions}

The studied we have performed in Nova and ElasticSearch has shown that the mean time to notify a bug in the system is higher than 10 months when the location of the bug introducing change is known. Whereas using SZZ to calculate this metric, the optimistic value is more than 5 months and the pessimistic value is more than 12 months. 

Our study also shows a weak negative correlation between the TTN and the EUB in Nova, which may indicates that a higher time to notify a fewer experience of the author. 

Once we have found this, it makes sense to explore, as future work, to which extent this happens in other projects with probably higher number of tickets and with a higher number metrics.

\section*{Acknowledgment}


We want to express our gratitude to Bitergia\footnote{\url{http://bitergia.com/}} for their open source tools to mining the repositories of the projects and the support they have provided when questions have arisen. Also, we acknowledge the Spanish Government, because some authors are funded in part by it, through project TIN2014-59400-R as well as the University the Victoria and University Rey Juan Carlos that have promoted the collaboration between Universities.

% trigger a \newpage just before the given reference
% number - used to balance the columns on the last page
% adjust value as needed - may need to be readjusted if
% the document is modified later
%\IEEEtriggeratref{8}
% The "triggered" command can be changed if desired:
%\IEEEtriggercmd{\enlargethispage{-5in}}

% references section

% can use a bibliography generated by BibTeX as a .bbl file
% BibTeX documentation can be easily obtained at:
% http://www.ctan.org/tex-archive/biblio/bibtex/contrib/doc/
% The IEEEtran BibTeX style support page is at:
% http://www.michaelshell.org/tex/ieeetran/bibtex/
%\bibliographystyle{IEEEtran}
% argument is your BibTeX string definitions and bibliography database(s)
%\bibliography{IEEEabrv,../bib/paper}
%
% <OR> manually copy in the resultant .bbl file
% set second argument of \begin to the number of references
% (used to reserve space for the reference number labels box)
%\begin{thebibliography}{1}

%\bibitem{IEEEhowto:kopka}
%H.~Kopka and P.~W. Daly, \emph{A Guide to \LaTeX}, 3rd~ed.\hskip 1em plus
%  0.5em minus 0.4em\relax Harlow, England: Addison-Wesley, 1999.

%\end{thebibliography}
\newpage

\bibliographystyle{abbrv}
\bibliography{sigproc} 


% that's all folks
\end{document}


